%%%%%%%%%%%%%%%%%%%%%%%%%%%%%%%%%%%%%%%%%%%%%%%%%%%%
%%%            Text
%%% Compile the master file! 
%%% Slides template: Antonio Machicao y Priemer
%%% Author: Antonio Machicao y Priemer
%%% Title: 
%%%%%%%%%%%%%%%%%%%%%%%%%%%%%%%%%%%%%%%%%%%%%%%%%%%%




%%%%%%%%%%%%%%%%%%%%%%%%%      
\begin{frame}
	\HUtitle
\end{frame}

\frame{
%\begin{multicols}{2}
	\frametitle{Content}
	\tableofcontents
	%[pausesections]
%\end{multicols}
	}

\section*{Introduction}
\begin{frame}[t]{Introductory Remarks} 
% [t] sets the frame alignment to the top, alternatves: c (center), b (bottom) -- can also be defined in the document class options 
	
	This example presentation aims to demonstrate basic functions in the \LaTeX -Beamer-Class. 
	For everything else you might want to check \dots  
	
	\begin{itemize}

		\item \LaTeX -Reader \hfill \citep{Freitag&MyP19a} 
		\item \href{https://www.linguistik.hu-berlin.de/de/staff/amyp/latex}{\LaTeX -Workshop} for Linguists \hfill \citep{MyP&Eberle19a} 
		\item \href{https://markov.htwsaar.de/tex-archive/macros/latex/contrib/beamer/doc/beameruserguide.pdf}{The Beamer Class User Guide} \hfill  \citep{Tantau_et_al_2024}
		
			\begin{itemize}
				\item and the \href{https://www.cpt.univ-mrs.fr/~masson/latex/Beamer-appearance-cheat-sheet.pdf}{Beamer Class Cheat Sheet} 
			\end{itemize}
	\end{itemize}
	
	\vspace{0.5cm}
	Otherwise \dots
	
		\begin{itemize}
			\item the \gqq{Style} file for this slide is stored in \gray{tex-styleHU/huberlin.sty} 
			\item packages are loaded in \gray{01packages-slides.tex}
			\item customised commands are defined in \gray{01commands-slides.tex} 
			\begin{itemize}
				\item \gray{01packages-slides.tex} and \gray{01commands-slides.tex} are loaded in the main .tex file via  \texttt{\textbackslash input\{\}} before \texttt{\textbackslash begin\{document\}} 
			\end{itemize}
		\end{itemize}
		
\end{frame}

	
	
%%%%%%%%%%%%%%%%%%%%%%%%%%
%%%%%%%%%%%%%%%%%%%%%%%%%%
\section{Introduction}
\frame{
%\begin{multicols}{2}
	\tableofcontents[currentsection,hidesubsections]
%\end{multicols}
}
%%%%%%%%%%%%%%%%%%%%%%%%%%

\begin{frame}{Introduction}
	
	\begin{itemize}
		\item A presentation serves as visual support for your talk,\\
		thus information and elements should be used sparsely to not distract the audience from what you have to say. 
		\item In this example presentation some of the most useful elements will be demonstrated. 
	\end{itemize}

\end{frame}

%%%%%%%%%%%%%%%%%%%%%%%%%%
%%%%%%%%%%%%%%%%%%%%%%%%%%
\section{Useful Presentation Elements}
\frame{
	%\begin{multicols}{2}
	%\frametitle{~}
	\tableofcontents[currentsection,hidesubsections]
	%\end{multicols}
}
%%%%%%%%%%%%%%%%%%%%%%%%%%


\subsection{Definitions \& Examples}
\begin{frame}{Definitions \& examples}
	
	\begin{itemize}
		\item First, you might want to start with some definitions, 
		the \texttt{\alert{block environment}}. 
		
		\begin{block}{Term}
			Definition of term
		\end{block}
		
		\vspace{.5cm}
		
		\item And some first examples demonstrating your discussed issue. 
		\begin{exe}
			\ex This is an example.
			\ex
			\begin{xlist}
				\ex another example
				\ex and another one
			\end{xlist}
		\end{exe}
		
	\end{itemize}
	

	
\end{frame}

\subsection{Multicols \& Minipages}

\begin{frame}{Multicols \& Minipages}
	
	\begin{itemize}
		\item The \texttt{\alert{multicols environment}} and \texttt{\alert{minipage environment}} can be used to split the space into two or more columns. 
	\end{itemize}

\vspace{.5cm}

\begin{minipage}[b][3cm][t]{0.35\textwidth}
	\begin{figure}
		\centering
		\includegraphics[scale=.5]{pictures/Georg-Cantor-1894}
		%\caption{Georg Cantor (ca.\ 1894)}
	\end{figure}
\end{minipage}\hfill
\begin{minipage}[b][3cm][t]{0.6\textwidth}
	\centering
	\begin{block}{Minipage}
		While the minipage is very useful for \textbf{images}, \textbf{tables}, \textbf{blocks}, and of course also text to put next to each other \dots
	\end{block}
\end{minipage}
\vspace{.5cm}

\begin{multicols}{3}
	\dots multicols might be better suited for 
	\begin{itemize}
		\item text
		\item ordered lists
		\begin{enumerate}
			\item or numbered lists
			\item \dots
		\end{enumerate}
		
		
	\end{itemize}
\end{multicols}
	
\end{frame}




%%%%%%%%%%%%%%%%%%%%%%%%%%
\subsection{Pauses \& simple Animation}


\begin{frame}[t]{Pauses \& simple Animation}

	\begin{itemize}
		\item point 1
		\item point 2
		\item point 3
	\end{itemize}

and then nothing for dramatic effect \dots 
\pause
\vspace{.5cm}

and now go on \dots 

%\begin{exe}
%	\item\only<2>{example 1}
%	\begin{xlist}
%			\only<3,4>{\item example 1}
%			\only<3,4>{\item modified example 1}
%	\end{xlist} 
%	
%	\only<4>{\item example 2}
%\end{exe}

\begin{exe}
	\item<2-> \only<2>{example 1}
	\begin{xlist}
		\item<3,4-> example 1
		\item<3,4-> modified example 1
	\end{xlist}
	\item<4->  example 2
\end{exe}

\end{frame}


\begin{frame}
	
	\begin{columns}
		
		\column[t]{.3\textwidth}
		
		ekngngiengnaei
		eargloanevpaenv
		aepvanervpnaevpö
		ekngngiengnaei
		eargloanevpaenv
		aepvanervpnaevpö
		
		\column[t]{.3\textwidth}
		
		ekngngiengnaei
		eargloanevpaenv
		aepvanervpnaevpö

		
		
		\column[t]{.3\textwidth}
		
		ekngngiengnaei
		eargloanevpaenv
		aepvanervpnaevpö	
		
		
	\end{columns}
	
\end{frame}
