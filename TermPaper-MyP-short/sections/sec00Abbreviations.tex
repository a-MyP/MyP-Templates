%%%%%%%%%%%%%%%%%%%%%%%%%%%%
%%%%%%%%%%%%%%%%%%%%%%%%%%%%
%% Package: acronym (see localpackages and documentation)

\section*{Abbreviations}
	\addcontentsline{toc}{section}{Abbreviations}
	\phantomsection%this allows hyperlink in ToC to work	


All abbreviations used in this work \gs{except the ones for glosses in examples} are listed below. For glossed examples, the norms and abbreviations supplied by the  \emph{Leipzig Glossing Rules} \citep[cf.][]{LeipzigGloss15a} were used.



\begin{multicols}{2}
\setlength{\columnseprule}{.5pt}
\begin{acronym}[VFKhaber]
	\acro{3}[\textit{3}]{third person (type)}
%
%A
	\acro{a.o.}{among others}
	\acro{AVM}{attribute-value-matrix}
%
%B
	\acro{BAG}{Bay Area Grammars}
%
%C
	\acro{CG}{Categorial Grammar}
%
%D
	\acro{dat}[\textit{dat}]{dative (type)}
	\acro{DTR}{daughter (attribute)}	
%
%E
	\acro{e.g.}{exempli gratia (=~for example)}
%
%F
	\acro{fem}[\textit{fem}]{feminine (type)}
%
%G
	\acro{GPSG}{Generalized Phrase Structure Grammar}
%
%H
	\acro{HFC}{head feature convention}	
	\acro{HPSG}{Head-Driven Phrase Structure Grammar}
%
%I
	\acro{i.e.}{id est (=~that is)}
	\acro{inf}[\textit{inf}]{infinite (type)}
	\acro{IPA}{International Phonetic Alphabet}	
%
%L
	\acro{LEX-DTR}{lexical daughter (attribute)}
	\acro{LFG}{Lexical Functional Grammar}
%
%M	
	\acro{masc}[\textit{masc}]{masculine (type)}
	\acro{mrs}[\textit{mrs}]{minimal recursion semantics (type)}
%
%N
	\acro{NC}{nominal complex}
	\acro{num}[\textit{num}]{number (type)}
%
%P
	\acro{PER}{person (attribute)}
	\acro{PHON}{phonology (attribute)}
	\acro{pl}[\textit{pl}]{plural (type)}
%
%Q
%
%R
	\acro{RELS}{relations (attribute)}	
%
%S
	\acro{synsem}[\textit{synsem}]{syntax-semantics (type)}
	\acro{SYNSEM}{syntax-semantics (attribute)}
%
%T
	\acro{TAG}{Tree Adjoining Grammar}
%
%U
%
%V
	\acro{vform}[\textit{vform}]{verb form (type)}
	\acro{VFORM}{verb form (attribute)}
	\acro{viz.}{videlicet ($=$ that is to say)}
%
%W	
%
%X
%
%Y
%
%Z	
\end{acronym}
\end{multicols}
