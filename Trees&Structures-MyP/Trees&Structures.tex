% !TEX TS-program = pdflatex
% !TeX encoding = UTF-8
% !TeX spellcheck = en_GB
%%%%%%%%%%%%%%%%%%%%%%%%%%%%%%
%%% Template for Trees & Structures
%%% Author: Antonio Machicao y Priemer
%%% Compile: PDF BibTeX PDF PDF
%%%%%%%%%%%%%%%%%%%%%%%%%%%%%%

\documentclass[11pt]{scrartcl}


%%%%%%%%%%%%%%%%%%%%%%
%%%%%%%%%%%%%%%%%%%%%%
%% PACKAGES:
\usepackage[utf8]{inputenc}
\usepackage[T3,T1]{fontenc}
\usepackage{lmodern}
\usepackage{xcolor}		%coloured elements
\usepackage[ngerman,english]{babel}

\usepackage{natbib}
	\setcitestyle{notesep={:~}}
	
\usepackage[noenc,safe]{tipa}
\usepackage{amssymb}
\usepackage[hidelinks]{hyperref}


%% TikZ-qtree
\usepackage{tikz-qtree}
	\usetikzlibrary{positioning}

%% Forest
\usepackage[linguistics,edges]{forest}

%% AVM
\usepackage{avm}

%%% Setting of avm (see LSP Guidelines)
%% http://langsci-press.org/public/downloads/LangSci_Guidelines.pdf
%	\avmfont{\sc}
%	\avmvalfont{\it}
\avmfont{\normalfont \scshape} 
\avmvalfont{\normalfont \itshape} 
%% command to fontify the type values of an avm 
\newcommand{\tpv}[1]{{\avmjvalfont #1}} 
%% command to fontify the type of an avm and avmspan it
\newcommand{\tp}[1]{\avmspan{\tpv{#1}}}


%%%%%%%%%%%%%%%%%%%%%%
%%%%%%%%%%%%%%%%%%%%%%
%%% Global tikzset

% has strange side effects
% allows for glosses in trees
\tikzset{every tree node/.style={align=left, anchor=north}}
\tikzset{every roof node/.append style={inner sep=0.1pt,text height=2ex,text depth=0.3ex}}
% alignment of the leave nodes:
% http://tex.stackexchange.com/questions/163056/aligning-several-trees-to-the-baseline
% seems not to work
\tikzset{every leaf node/.append style={text depth=0pt}}
%%%%%%%%%%%%%%%%%%%%%%


%%%%%%%%%%%%%%%%%%%%%%
%%%%%%%%%%%%%%%%%%%%%%
%%% Global forestset
\forestset{
	%% makes the tree appear lighter in color:
	background tree/.style={for tree={text opacity=0.2,draw opacity=0.2,edge={draw opacity=0.2}}},
	%% allows for branch-offs even when there's no node (this set needs to be written in the position of the "missing" node aswell):
	empty nodes/.style={
		delay={where content={}{shape=coordinate,for parent={for
					children={anchor=north}}}{}}},
	%% aligns every terminal node with the tree's lowest terminal node:
	bottom word/.style={for tree={parent anchor=south, child anchor=north,align=center,base=bottom,
			where n children=0{tier=word}
			{}}},
	%% edges appear as arrows	
	%% https://tex.stackexchange.com/questions/187565/downwards-arrows-in-forest
	edgy/.style={for tree={edge={->} }},
}
%%%%%%%%%%%%%%%%%%%%%%



%%%%%%%%%%%%%%%%%%%%%%
%%%%%%%%%%%%%%%%%%%%%%
%% METADATA:

\title{Trees \& Structures}

\subtitle{(avm, forest, tikz)}

\author{Antonio Machicao y Priemer\\
	\normalsize \url{https://hu.berlin/aMyP}%\\
%	\small \href{mailto:machicao.y.priemer@hu-berlin.de}{machicao.y.priemer@hu-berlin.de}
	}

%\date{13. Juni 2013}


%%%%%%%%%%%%%%%%%%%%%%
%%%%%%%%%%%%%%%%%%%%%%
\begin{document}

\maketitle

\tableofcontents

%\nocite{Nordhoff&Co}

\clearpage


%%%%%%%%%%%%%%%%%%%%%%
%%%%%%%%%%%%%%%%%%%%%%
\section{Notes}

For this file you will need the following packages:

\begin{itemize}
	
	\item Fontenc package with T1 and T3 option:
	
		\verb|\usepackage[T3,T1]{fontenc}|
	
	\item Xcolor package for colored elements in trees:
	
		\verb|\usepackage{xcolor}|
	
	\item Tipa package with no encoding and safe option:
	
		\verb|\usepackage[noenc,safe]{tipa}|
	
	\item TikZ-qtree package with the positioning library:
	
		\verb|\usepackage{tikz-qtree}|
	
		\verb|\usetikzlibrary{positioning}|
	
	\item Forest package with linguistics option:
	
		\verb|\usepackage[linguistics]{forest}|
	
	\item AVM package (the one in this folder%
	%
	\footnote{There are many different versions of the package avm on the internet. They have different settings but the same name. So if you are using a different avm package, it could be the case, that you get some errors.}%
	%
	)):
	
		\verb|\usepackage{avm}|
	
\end{itemize}


If the settings (e.g. \texttt{forestset} or \texttt{tikzset}) are used outside of the forest or tikz-picture environment (see the code in the tex-file of this document) then they apply globally, i.e.\ for all following trees. If they are only used inside of an environment (i.e. after \verb|\begin{forest}| or \verb|\begin{tikzpicture}|), their effect only lasts until the environment is closed again (see code below).

For further information on \LaTeX , forest, TikZ, and tipa, see \citet{Freitag&MyP15a,VandenWyngaerd16a,Zivanovic17a,Cremer11a,Tantau13a,Rei04a}. 

This file has been compiled with PDF-\LaTeX .


%\clearpage

%%%%%%%%%%%%%%%%%%%%%%
%%%%%%%%%%%%%%%%%%%%%%
\section{AVM}

\subsection{Two examples with different commands}

First example (see code):
\vspace{-1cm}
\begin{center}
	\begin{avm}
\avml

\avml \\
	\[{}subcat \< \avml \hfil NP \hfil \\[-1ex] 
	\[{}case&nom\\
	ind&\@1\]
\avmr, 
\avml 
	\hfil NP \hfil \\[-1ex] 
	\[{}case&acc\\
	ind&\@2\] 
\avmr \>\] 
\avmr
\raisebox{-.25cm}{$\Longrightarrow$}
\avml \\ 
	\[{}subcat \<\avml \hfil NP \hfil \\[-1ex] 
	\[{}case&nom\\
	ind&\@3\] 
\avmr, 
\avml 
	\hfil NP \hfil \\[-1ex] 
	\[{}case&dat
	\\ind&\@1\] 
\avmr, 
\avml 
	\hfil NP \hfil \\[-1ex] 
	\[{}case&acc\\ind&\@2\] 
\avmr \>\] 
\avmr 

\avmr 
\end{avm}
\end{center}

\bigskip

\noindent Second example (see code):
\vspace{-.5cm}
\begin{center}
	\begin{avm}
		\[subcat \< 
			\avml
				\hfil NP \hfil \\[-1ex] 
				\[case&nom\\
				ind&\@1\]
			\avmr, 
			\avml 
				\hfil NP \hfil \\[-1ex] 
				\[case&acc\\
				ind&\@2\] 
			\avmr \>
		\] 
$\Longrightarrow$ 
		\[subcat \<
			\avml 
				\hfil NP \hfil \\[-1ex] 
				\[case&nom\\
				ind&\@3\] 
			\avmr, 
			\avml 
				\hfil NP \hfil \\[-1ex] 
				\[case&dat\\
				ind&\@1\] 
			\avmr, 
			\avml 
				\hfil NP \hfil \\[-1ex] 
				\[case&acc\\
				ind&\@2\] 
			\avmr \>
		\]
\end{avm}
\end{center}


\subsection{Lexical Rule}

%%%%%%%%%%%%%%%%%%%%%%%%%%%%%%
%% BEGIN Figure 
\begin{figure}[htbp]
	\begin{avm}
		\scriptsize
		\[cont$|$rels 
		\@{8}
		\, $\oplus$ 
		\tpv{nelist}
		\\
		\tpv{alt-psych-v-lxm}							
		\]
	\end{avm}
	%	
	\begin{minipage}[t][][t]{.03\textwidth}
		\vspace{.05cm}
		$\mapsto$
	\end{minipage}
	%	
	\flushright	
	%	
	\begin{avm}
		\scriptsize
		\[cat$|$arg-st \< NP{[\tpv{str}]}$_{\@{5}}$ , NP{[\tpv{str}]}$_{\@{1}}$\>\\
		%%
		cont
		\[
		ind \@{4}\\
		%%
		rels 
		\@{8}
		\<
		\[ 
		arg0 \@{0}\\
		\tpv{pred}
		\] 
		{,}
		\[ 
		arg0 \@{1} \\
		\tpv{exp}
		\]
		\>\, $\oplus$ 
		\<  
		\[
		arg0 \@{4} \tpv{hpng}\\
		arg1 \@{0}\\
		\tpv{begin-pred}
		\]{,}
		\[ 
		arg0 \@{5} \\
		arg1 \@{4}\\
		\tpv{csr}
		\]
		\> 
		\]\\
		\tpv{cause-psych-v-lxm}		    
		\]
	\end{avm}
	
	\caption{LR for case alternation for \emph{alt-psych-v-lxm} \citep{MyP&Co18a}}
	\label{fig:LRSpa}
\end{figure}

%% END Figure 
%%%%%%%%%%%%%%%%%%%%%%%%%%%%%%

\clearpage 

%%%%%%%%%%%%%%%%%%%%%%
%%%%%%%%%%%%%%%%%%%%%%
\subsection{Forest-Tree with AVM}

%%%%%%%%%%%%%%%%%%%%%%%%%%%%%%
\begin{figure}[htbp]
	\centering
	\scalebox{.95}{\begin{forest}
		[N$'_{ii}$\\
		\begin{avm}
			\begin{scriptsize}
				\[cont & 
				\[
				ind \@{1}
				\[per  & 3 \\
				num  & sg \\
				gend & masc
				\]\\
				rels 
				\< \@{3}{,} \@{7}{,} \@{4}{,} \@{5} \> \\
				\tpv{mrs}
				\]\\
				\tp{head-complement-structure}		
				\]		
			\end{scriptsize}
		\end{avm} 
		[N$^{0}_{ii}$\\ 
		\begin{avm}
			\begin{scriptsize}
				\[
				cont & 
				\[ind \@{1}
				\[per  & 3 \\
				num  & sg \\
				gend & masc
				\]\\
				rels 
				\<
				\@{3}
				\[inst \@{1} \\
				result \@{6} \\
				\tpv{win\_result}
				\]{,}
				\@{7}
				\[event \@{6} \\
				th \@{2} \\
				\tpv{win}
				\]
				\> \\
				\tpv{mrs}
				\]\\
				\tp{word}	
				\]		
			\end{scriptsize}
		\end{avm} 
		[Gewinn \\ win]]
		[NP$_{i}$\\ 
		\begin{avm}
			\begin{scriptsize}
				\[cont &
				\[
				ind \@{2}
				\[per  & 3 \\
				num  & sg \\
				gend & fem
				\]\\
				rels 
				\< 
				\@{4}
				\[arg \@{2}\\
				\tpv{def}
				\]{,} 
				\@{5}
				\[inst \@{2} \\
				\tpv{world\_}\\
				\tpv{championship}
				\] 
				\> \\
				\tpv{mrs}
				\]\\
				\tp{head-specifier-structure}		
				\]			
			\end{scriptsize}
		\end{avm}
		[der WM \\ of the World Championship]]
		]	
		]
	\end{forest} }
	\caption{Illustration of the Semantics Principle \citep{MyP17e}}
	\label{fig:SemP1}
\end{figure}
%%%%%%%%%%%%%%%%%%%%%%%%%%%%%%


%%%%%%%%%%%%%%%%%%%%%%
%%%%%%%%%%%%%%%%%%%%%%
\section{Forest-Trees (Basics)}


%%%%%%%%%%%%%%%%%%%%%%
%%%%%%%%%%%%%%%%%%%%%%
\subsection{Simple small tree with bar over X, no bottom alignment}

\begin{center}
\begin{forest}
[CP
	[DP$_2$
		[$\overline{\mbox{D}}$
			[D$^0$
				[sie]
			]
		]
	]	
	[$\overline{\mbox{C}}$
		[C$^0$
			[besucht$_1$]
		]
		[IP
			[$t_2$]
			[$\overline{\mbox{I}}$
				[VP
					[$\overline{\mbox{V}}$
						[DP
							[$\overline{\mbox{D}}$
								[D$^0$
									[ihn]
								]
							]
						]
						[V$^0$
							[$t_1$]
						]
					]
				]
				[I$^0$
					[$t_1$]
				]
			]
		]
	]
]
\end{forest}
\end{center}


%%%%%%%%%%%%%%%%%%%%%%
%%%%%%%%%%%%%%%%%%%%%%
\subsection{Trinary branching, prime instead of bar, bottom alignment}

\begin{center}
	\begin{forest}
		bottom word
		[VP
			[DP [John]]
			[V$'$
				[V [sent]]
				[DP [Mary]]
				[DP
					[D[a]]
					[NP[letter]]
				]
			]
		]
	\end{forest}
\end{center}


%%%%%%%%%%%%%%%%%%%%%%
%%%%%%%%%%%%%%%%%%%%%%
\subsection{Bottom alignment, roof, traces}

\begin{figure}[h!]
\centering	
\begin{forest}
bottom word
[CP
[C$'$
	[C$^0$[{[kenn-$_{\tiny j}$ -t]}$_{\tiny k}$]]
	[IP
		[NP[jeder]]
		[I$'$
			[VP
				[V$'$
					[NP[diesen Mann ,roof]]
					[V$^0$[\_$_{\tiny j}$]]]]
			[I$^0$[\_$_{\tiny k}$]]]]]]
\end{forest}
\caption{CP Structure in \citet[107]{MuellerS19a}}
\end{figure}

\clearpage


%%%%%%%%%%%%%%%%%%%%%%
%%%%%%%%%%%%%%%%%%%%%%
\subsection{Bottom alignment with \texttt{tier=word} and empty nodes}

The command \texttt{, tier=word} aligns every node with this command to the lowest node that has the command.

\begin{center}
	\begin{forest}
		empty nodes
		[{CP}
		[{}[{AdvP} ,tier=word [{heute} ,roof]]]
		[{C$'$}
		[{}[{C$^0$} ,tier=word [{kauft$_1$}]]]
		[{IP}
		[{}[{DP} ,tier=word [{Peter} ,roof]]]
		[{I$'$}
		[{VP}
		[{DP} ,tier=word [{einen Wagen} ,roof]]
		[{V$^0$} ,tier=word [{$t_1$}]]]
		[{}[{I$^0$} ,tier=word [{$t_1$}]]]]]]]
	\end{forest}
\end{center}


%%%%%%%%%%%%%%%%%%%%%%
\subsection{Big tree -- resized, with phantom nodes}


\begin{center}
\resizebox{\textwidth}{!}{%
\begin{forest}
	bottom word
[CP
	[C$^0$[weil]]
	[TopP
		[DP$_{\tiny j}$[diese Sonate]]
		[SubjP
			[DP$_{\tiny i}$[der Mann]]
			[ModP
				[AdvP[wahrscheinlich]]
				[ObjP
					[DP$_{\tiny j}$[diese Sonate]]
					[NegP
						[AdvP[nicht]]
						[AspP
							[AdvP[oft]]
							[MannP
								[AdvP[gut]]
								[AuxP
									[VP$_{\tiny k}$[gespielt]]
									[Aux+
										[Aux[hat]]
										[vP
											[DP$_{\tiny i}$
												[,phantom]]
											[VP$_{\tiny k}$
												[V
													[,phantom]]
												[DP$_{\tiny j}$
													[,phantom]]]]]]]]]]]]]]
\end{forest}%
}
\end{center}


%%%%%%%%%%%%%%%%%%%%%%
%%%%%%%%%%%%%%%%%%%%%%
\subsection{Two trees and arrow}

\begin{center}
	\begin{forest}
		[S, for tree={parent anchor=south, child anchor=north}
		[NP [John] ]
		[VP
		[V [loves] ]
		[NP [Mary] ] 
		]]
	\end{forest}
\qquad  \raisebox{-2cm}{$\rightsquigarrow$}  \qquad
	\begin{forest}
		[S, for tree={parent anchor=south, child anchor=north}
		[NP[Mary]]
		[VP
		[Aux[is, tier=word]]
		[V[loved, tier=word]]
		[PP
		[P[by, tier=word]]
		[NP[John, tier=word]]]]]
	\end{forest}
\captionof{figure}{Transformation \citep[149, 85]{MuellerS19a}}
\end{center}


%%%%%%%%%%%%%%%%%%%%%%
%%%%%%%%%%%%%%%%%%%%%%
\subsection{Node with circle}

\begin{center}
\begin{forest} 
[VP,circle,draw
	[DP]
	[V$'$ [V]
		[DP]]
]
\end{forest}
\end{center}


%%%%%%%%%%%%%%%%%%%%%%
%%%%%%%%%%%%%%%%%%%%%%
\subsection{Two nodes marked with ellipse}

Change the parameters in \texttt{node} to fit the nodes inside the ellipse.

\begin{center}
	\begin{forest} 
		[VP,
		[DP, name=DP]
		[V$'$ 
			[V, name=V
			]
		[DP]]
		]{
			\node[ellipse, draw=blue, rotate=26, fit=(DP) (V), scale=0.67, inner sep=1.5pt, yshift=0.1cm, xshift=0cm, ] {};
		}
	\end{forest}
\end{center}


\noindent Code taken from: \href{https://tex.stackexchange.com/questions/355365/drawing-an-ellipse-around-an-edge-in-forest}{https://tex.stackexchange.com/questions/355365/drawing-an-ellipse-around-an-edge-in-forest}


%%%%%%%%%%%%%%%%%%%%%%
%%%%%%%%%%%%%%%%%%%%%%
\subsection{Coloured rectangle}

\begin{center}
\begin{forest}
[CP
	[DP$_1$]
	[\dots
		[,phantom]
		[VP,tikz={\node [draw,red,fit=()] {};}
			[DP$_2$]
			[V'
				[V]
			[DP$_3$]
]]]]
\end{forest}
\end{center}


%%%%%%%%%%%%%%
%%%%%%%%%%%%%%%%%%%%%%
\subsection{Forest-Trees with edges and crossing edges}


%% Needs options "linguistics" and "edges"
%% \usepackage[linguistics,edges]{forest}
\begin{centering}

\begin{forest}
	for tree={grow=south,delay={where content={}{shape=coordinate}{}}}, 
	forked edges, l sep=2em
	[S,s sep=1cm
	[Kn, l sep=2em
	[Er]
	]
	[V, l sep=2em
	[sieht]
	[ihn, name=ihn, edge=white]
	[an]
	]
	[Kn, name=kn, l sep=2em
	[, edge=white]
	]
	]{
		\draw[,-] (kn.south)--(ihn.north);
	}
\end{forest}

\end{centering}


\bigskip
\noindent Lengthening the edges:

\begin{centering}
	
\begin{forest}
	for tree={grow=south,delay={where content={}{shape=coordinate}{}}}, 
	forked edges
[S, before drawing tree={x-=3mm},
	[Kn, before drawing tree={y-=13mm},
		[Er, l+=13mm, tier=word]
	]
	[V
		[sieht, tier=word]
		[an, before drawing tree={x+=18mm}, tier=word]
	]
	[Kn, before drawing tree={y-=13mm},
		[ihn, tier=word]
	]
]
\end{forest}

\end{centering}


%%%%%%%%%%%%%%
%%%%%%%%%%%%%%%%%%%%%%
\section{Forest-Trees with arrows}


%%%%%%%%%%%%%%%%%%%%%%
\subsection{Movement and advice, with phantom node}

\begin{center}
	\begin{forest}
		[CP
		[DP,name=spec CP]
		[\dots
		[,phantom]
		[VP
		[DP]
		[V'
		[V]
		[DP,draw] {
			\draw[->,dotted] () to[out=south west,in=south] (spec CP);
			\draw[<-,red] (.south east)--++(0em,-4ex)--++(-2em,0pt)
			node[anchor=east,align=center]{This guy\\has moved!};
		}
		]]]]
	\end{forest}
\captionof{figure}{CP with arrows \citep[6, 8]{Zivanovic17a}}
\end{center}


%%%%%%%%%%%%%%%%%%%%%%
\subsection{With different arrows}

%%%%%%%%%%%%%%%
\begin{figure}[htbp]
	\centering
	
	\begin{minipage}[c][][c]{.33\textwidth}
		\centering
		\begin{forest}
			[X\\phrase ,name=Phrase
			[Z\\argument ,name=Argument]
			[Y\\head ,name=Head]
			]
			\draw[->,dotted] (Argument) to[out=north,in=north] (Head);
			\draw[->] (Head) to[out=south,in=south] (Argument);
			\draw[->,dashed] (Head) to[out=east,in=east] (Phrase);
		\end{forest}\hspace{1cm}
	\end{minipage}
	%%
	\begin{minipage}[c][][c]{.01\textwidth}
		~
	\end{minipage}
	%%
	\begin{minipage}[c][][c]{.62\textwidth}
		\centering
		\begin{tabular}[b]{ll}
			\tikz[baseline]\draw[dashed](0,1ex)--(1,1ex); & HFP , ValP \& SemP (\textsc{ind})\\
			\tikz[baseline]\draw(0,1ex)--(1,1ex); & selection through \textsc{val} (\textsc{subj}, \textsc{comps})\\
			\tikz[baseline]\draw[dotted](0,1ex)--(1,1ex); & incorporation of the value of \textsc{ind} into \textsc{rels}\\
		\end{tabular}
	\end{minipage}
	\caption{Head-argument relation \citep{MyP17e}}
	\label{fig:Tree-Arg2}
\end{figure}
%%%%%%%%%%%%%%%


%%%%%%%%%%%%%%%
\begin{figure}[htbp]
	\centering
	\begin{minipage}{.35\textwidth}
		\centering
		\begin{forest}
			[X\\phrase ,name=Phrase
			[W\\adjunct ,name=Adjunct]
			[Y\\head ,name=Head]
			]
			\draw[->,dotted] (Head) to[out=north,in=north] (Adjunct);
			\draw[->] (Adjunct) to[out=south,in=south] (Head);
			\draw[->,dashed] (Head) to[out=east,in=east] (Phrase);
			\draw[->,dashed,ultra thick] (Adjunct) to[out=west,in=west] (Phrase);
		\end{forest}\hspace{1cm}
	\end{minipage}
	%%
	\begin{minipage}{.02\textwidth}
		~
	\end{minipage}
	%%
	\begin{minipage}{.51\textwidth}
		\centering
		\begin{tabular}[b]{ll}
			\tikz[baseline]\draw[dashed](0,1ex)--(1,1ex); & HFP \& ValP \\
			
			\tikz[baseline]\draw[dashed,ultra thick](0,1ex)--(1,1ex); & SemP (\textsc{cont})\\
			
			\tikz[baseline]\draw(0,1ex)--(1,1ex); & selection through \textsc{mod}\\
			
			\tikz[baseline]\draw[dotted](0,1ex)--(1,1ex); & incorporation of the value of \textsc{cont}\\
		\end{tabular}
		
	\end{minipage}
	\caption{Head-adjunct relation \citep{MyP17e}}
	\label{fig:Tree-Adj2}
	
\end{figure}
%%%%%%%%%%%%%%%%%%%%%%


\begin{center}
\begin{forest}
	[NP, name=N2
	[\textsc{Det} [die]]
	[N$'$, name=N1
	[N$^0$, name=N0 [Behandlung]]
	[NP [des Patienten, roof]]
	]
	]{
		\draw[->,dashed] (N0) to[out=west,in=west] (N1);
		\draw[->,dashed] (N1) to[out=east,in=east] (N2);
	}
\end{forest}
\captionof{figure}{Projection of head features \citep{MyP18b}}\label{Abb3}
\end{center}


%%%%%%%%%%%%%%%%%%%%%%
\subsection{Tree with different arrows and coloured boxes}

\begin{center}

\begin{forest}
	[CP, fill=yellow,draw,circle,
	[DP$_{1}$, draw=red, name=T12
	[D$'$
	[D$^0$ [$\emptyset$]]
	[NP [Syntaktiker, roof]]
	] 
	]
	[C$^{\prime}$
	[C$^{0}$ [zeichnen$_{2}$, draw=blue, name=C0]]
	[TP, fill=yellow,draw,circle, [$t_{1}$, draw=red, name=T11]
	[T$^{\prime}$
	[VP, fill=red 
	[V$^{\prime}$ [DP [Bäume,roof]]
	[V$^{0}$ [$t_{2}$, draw=blue, name=V0]]]]
	[T$^{0}$ [$t_{2}$, draw=blue, name=T0]]]]]
	]
	\draw[->] (V0) to[out=east, in=east](T0);
%		\draw[->] (T0) to[out=south east, in=west](C0);
%		\draw[->] (T11) to[out=south west, in=west](T12);
\end{forest}

\end{center}

\clearpage


%%%%%%%%%%%%%%%%%%%%%%
\section{Forest-Trees with adjusted roofs for glosses and bottom alignment}

Taken from: \url{http://tex.stackexchange.com/questions/167978/smaller-roofs-for-forest}


%%%%%%%%%%%%%%%%%%%%%%
%%%%%%%%%%%%%%%%%%%%%%
\subsection{The default behaviour}

\begin{center}
\begin{forest}
bottom word
[NP
  [Det [das\\the] ]
  [N$'$,s sep=20pt
    [N$'$,s sep=15pt
      [N [Bild\\picture] ]
      [PP [vom Gleimtunnel\\ of.the Gleimtunnel,roof ] ] ] 
    [PP [im Gropiusbau\\ in.the Gropiusbau,roof ] ] ] ]
\end{forest}
\end{center}


%%%%%%%%%%%%%%%%%%%%%%
%%%%%%%%%%%%%%%%%%%%%%
\subsection{Hiding the wider text}

\newcommand{\HideWd}[1]{%
	\makebox[0pt]{#1}%
}
%%%%%%%%%%%%%

\begin{center}
\begin{forest}
bottom word
[NP
  [Det [das\\the] ]
  [N$'$
    [N$'$
      [N [Bild\\picture] ]
      [PP [vom Gleimtunnel\\ \HideWd{of.the Gleimtunnel},roof ] ] ] 
    [PP [im Gropiusbau\\ \HideWd{in.the Gropiusbau},roof ] ] ] ]
\end{forest}
\end{center}


\clearpage


%%%%%%%%%%%%%%%%%%%%%%
%%%%%%%%%%%%%%%%%%%%%%
\subsection{Tree with arrows avoiding nodes (with corrections)}

\begin{center}

\begin{forest}
[CP
	[C$'$
		[C$^0$ [sah$_1$, name=verb2]]
		[IP
			[DP$_3$, name=object [den Mann, roof]]
			[IP
				[DP$_2$, name=subject 
					[die Frau,roof]
				]
				%%
				[I$'$
					[VP
						[$t_2$, name=bsubject]
						[V$'$
							[$t_3$, name=bobject]
								[V$^0$ [$t_1$,name=bverb]]
						]
					]
					[I$^0$ , [$t_1$, name=verb1]]
				]
			]
		]
	]
]	
%\draw[->,dotted] (bsubject) to[out=west, in=west] (subject);
\draw[->,dashed] (bsubject.west) .. controls (-0.6,-7) and (0,-5.5) .. (subject.west);
\draw[->,dashed] (bobject.west) .. controls (-1,-8) and (-2.7,-5.3) ..  (object.west);
\draw[->,dashed] (bverb) to[out=east, in=east] ($(verb1)-(-0.2,.1)$);
\draw[->,dashed] ($(verb1.east)+(0,.1)$) .. controls (9.5,-11) and (-7,-13) ..  (verb2.west);
\end{forest}

\end{center}

\noindent Check also:\\ \href{https://tex.stackexchange.com/questions/352873/drawing-lines-or-arrows-along-node-pathes-with-forest/353341\#353341}{https://tex.stackexchange.com/questions/352873/drawing-lines-or-arrows-along-node-pathes-with-forest/353341\#353341}



%%%%%%%%%%%%%%%%%%%%%%
%%%%%%%%%%%%%%%%%%%%%%
\subsection{Hiding the wider text and correcting the separation}

\begin{center}
\begin{forest}
bottom word
[NP
  [Det [das\\the] ]
  [N$'$,s sep=20pt
    [N$'$,s sep=15pt
      [N [Bild\\picture] ]
      [PP [vom Gleimtunnel\\ \HideWd{of.the Gleimtunnel},roof ] ] ] 
    [PP [im Gropiusbau\\ \HideWd{in.the Gropiusbau},roof ] ] ] ]
\end{forest}
\end{center}


%%%%%%%%%%%%%%%%%%%%%%
\section{Some other trees for linguistics}
%%%%%%%%%%%%%%%%%%%%%%

\subsection{Language architecture}

This tree uses the forest styles \texttt{bottom word} and \texttt{edgy} defined in the preamble of this document.

\begin{center}
\begin{forest}
bottom word, edgy,
[D-structure
  [S-structure, edge label={node[midway,right]{\hspace{5mm}move-$\alpha$}}
  	[Deletion rules{,}\\Filter{,} phonol. rules
		[Phonetic\\Form (PF)]]
	[Anaphoric rules{,}\\rules of quantification and control
		[Logical\\Form (LF)]]]]
\end{forest}
\captionof{figure}{T-Modell \citep[88]{MuellerS19a}}
\end{center}


%%%%%%%%%%%%%%%%%%%%%%
%%%%%%%%%%%%%%%%%%%%%
\subsection{Structures of complex words}

\begin{center}
	
\scalebox{.8}{
	\begin{forest}
		[N, name=N1
			[N, name=N2
				[N, name=N3
					[A [Laut] ]
				[N, name=N4
					[V [sprech] ]
					[N\textsuperscript{af} [er] ]
				]
			]
			[N, name=N5
				[V, name=V1
					[Part [an] ]
					[V [weis] ]
				]
				[N\textsuperscript{af} [ung] ]
			]
		]
		[Fl [en] ]
		]
		{
			\draw[<-] (N1.west)--++(180:12.6em)
			node[anchor=east,align=center]{Flexion (KEIN Wortbildungsporzess)};
			\draw[<-] (N2.west)--++(180:8.9em)
			node[anchor=east,align=center]{Determinativkompositum};
			\draw[<-] (N3.west)--+(180:3.8em)
			node[anchor=east,align=center]{Determinativkompositum};
			\draw[<-] (N4.west)--+(-2em,0pt)--++(-2em,-1.5ex)--++(-4em,0pt)
			node[anchor=east,align=center]{Derivation(ssuffigierung)};
			\draw[<-] (N5.east)--++(0.5em,0pt)--++(0em,-3ex)--++(2em,0pt)
			node[anchor=west,align=center]{Derivation(ssuffigierung)};
			\draw[<-] (V1.east)--++(1.9em,0pt)--++(0em,-12ex)--++(2.5em,0pt)
			node[anchor=west,align=center]{Partikelverbbildung};
		}	
	\end{forest}
}
\captionof{figure}{Word structure \citep{MyP19a}}
\end{center}


%%%%%%%%%%%%%%%%%%%%%%
%%%%%%%%%%%%%%%%%%%%%%

\subsection{Structures of syllables}

\begin{center}
	\begin{forest}
	[,phantom
		[$\sigma$
			[O
				[x, tier=word
					[\textipa{l}]
				]
			]
			[R
				[N
					[x, tier=word
						[\textipa{\t{aU}}, name=aU]
					]
					[x, name=x]
				]  		
				[K 
					[x
						[\textipa{t}]
					]
				]
			]
		]
		[$\sigma$
			[O
				[x, tier=word
					[\textipa{S}]
				]
				[x, tier=word
					[\textipa{p}]
				]
				[x, tier=word
					[\textscr ]
				]
			]
			[R
				[N
					[x
						[\textipa{E}]
					]
				]
				[K
					[x, name=xc
						[\textipa{\c{c}}]
					]
				]
			]
		]
		[$\sigma$
			[O, name=O]
			[R
				[N
					[x
						[\textipa{5}]
					]
				]
				[K]
			]
		]
	]
\draw[black] (aU.north)--(x.south);
\draw[black] (O.south)--(xc.north);
	\end{forest}

\captionof{figure}{Phonetic structure \citep{MyP19a}}
\end{center}


\clearpage


The following style can be obtained using the forestset ``GP1'' which is already provided by the linguistics option of \texttt{forest}.

\begin{center}
	\begin{forest} GP1 
	[,phantom
		[$\sigma$
			[O
				[x, tier=word [\textipa{l}]]]
			[R
				[N
					[x, tier=word [\textipa{\t{aU}}, name=aU]]
					[x, name=x]]  		
				[K 
					[x	[\textipa{t}]]]
			]
		]
		[$\sigma$
			[O
				[x, tier=word [\textipa{S}]]
				[x, tier=word [\textipa{p}]]
				[x, tier=word [\textscr ]]]
			[R
				[N
					[x [\textipa{E}]]
				]
				[K [x, name=xc [\textipa{\c{c}}]]
				]
			]
		]
		[$\sigma$
			[O, name=O]
			[R
				[N
					[x [\textipa{5}]]
				]
				[K]
			]
		]
	]
	\draw[black] (aU.north)--(x.south);
	\draw[black] (O.south)--(xc.north);
	\end{forest}
\captionof{figure}{Phonetic structure \citep{MyP19a}}
\end{center}


%%%%%%%%%%%%%%%%%%%%%%
\subsection{Sonority Profiles with TikZ}


\begin{center}

	\begin{tikzpicture}[scale=0.5]
	\draw[black] (-1,0) -- (6.5,0) ; % x axis
	\draw[black] (-1,0) -- (-1,6.5); % y axis
	\node at (-2,1) {Plos.};
	\node at (-2,2) {Fric.};
	\node at (-2,3) {Nas.};
	\node at (-2,4) {\textipa{/l/}};
	\node at (-2,5) {\textipa{/\textscr /}};
	\node at (-2,6) {Vow.};
	\draw[black] (0,2) -- (1,1) -- (2,5) -- (3,6) -- (4,2) -- (5,6) -- (6,3);
	\node at (0,-1) {\strut \textipa{S}};
	\node at (1,-1) {\strut \textipa{p}};
	\node at (2,-1) {\strut \textipa{\textscr}};
	\node at (3,-1) {\strut \textipa{E}};
	\node at (4,-1) {\strut \textipa{\c{c}}};
	\node at (5,-1) {\strut \textipa{@}};
	\node at (6,-1) {\strut \textipa{n}};
	\fill (0,2) circle [radius=3pt];
	\fill (1,1) circle [radius=3pt];
	\fill (2,5) circle [radius=3pt];
	\fill (3,6) circle [radius=3pt];
	\fill (4,2) circle [radius=3pt];
	\fill (5,6) circle [radius=3pt];
	\fill (6,3) circle [radius=3pt];
	\end{tikzpicture}

\captionof{figure}{Sonority profile \citep{MyP19a}}
\end{center}


%%%%%%%%%%%%%%%%%%%%%%
%%%%%%%%%%%%%%%%%%%%%%
\subsection{Tikz-tree: Typology}

\begin{center}
\begin{tikzpicture}
%%%%%%%%%%%%%
%%% Tikz set
\tikzset{edge from parent/.style={draw,edge from parent path={(\tikzparentnode.south)-- +(0,-8pt)-| (\tikzchildnode)}}}

\Tree [.ZZ
 [.Bax
 [.X
 [.Y [.A ] [.B ] ]
 [.Z [.C ] [.D ] ] ]
 [.F
 [.M [.E ] [.F ] ]
 [.G [.G ] [.H ] ] ] ]
 [.A
 [.B
 [.S  [.I P R T V U ] [.J ] ]
 [.I  [.K ] [.L ] ] ]
 [.M
 [.L  [.M ] [.N ] ]
 [.A  [.O ] [.P ] ] ] ] ] ]
\end{tikzpicture}
\end{center}


%%%%%%%%%%%%%%%%%%%%%%
%%%%%%%%%%%%%%%%%%%%%%
\subsection{Forest-tree: Typology}

\begin{center}
 \begin{forest}
[ZZ
	[Bax
		[X
			[Y [A] [B] ]
			[Z [C] [D] ] ]
		[F
			[M [E] [F] ]
			[G [G] [H] ] ] ]
		[A
			[B
				[S  [I [P][R][T][V][U]] [J] ]
				[I  [K] [L] ] ]
			[M
				[L  [M] [N] ]
				[A  [O] [P] ] ] ] ]
 \end{forest}
\end{center}


%%%%%%%%%%%%%%%%%%%%%%
%%%%%%%%%%%%%%%%%%%%%%
\subsection{Forest Sets: rectangles}



%%%%%%%%%%%%%
\begin{center}
	
\pgfkeys{/pgf/inner sep=0.6666em}
\begin{forest}
sn edges,
[Set 3,draw 
	[Set 1,draw 
		[German]
		]
	[Set 2,draw 
		[Dutch]]
]{\draw[black] (.west) node[anchor=east,align=center]{ArgStr\\SOV\\V2\\VC};}
\end{forest}

\end{center}




%%%%%%%%%%%%%
\begin{center}
	
\pgfkeys{/pgf/inner sep=0.6666em}
\begin{forest}
sn edges,
[Set 5,draw
[Set 4,draw 
	[Set 1,draw 
		[German]]
	[Set 2,draw 
		[Dutch]]
]{\draw[black] (.west)
node[anchor=east,align=center]{SOV\\VC};}
[Set 6,draw
	[Danish]]]
{\draw[black] (.west)
node[anchor=east,align=center]{ArgStr\\V2};}
\end{forest}
\end{center}
%%%%%%%%%%%%%


%%%%%%%%%%%%%
\begin{center}
\pgfkeys{/pgf/inner sep=0.6666em}
\begin{forest}
sn edges,
[Set 8,draw
	[Set 7,draw
		[Set 4,draw 
			[Set 1,draw 
				[German]]
			[Set 2,draw 
				[Dutch]]
		]{\draw[black] (.west)
node[anchor=east,align=center]{SOV\\VC};}
		[Set 6,draw
			[Danish]]
	]{\draw[black] (.west)
node[anchor=east,align=center]{V2};}
	[Set 11,draw
		[Set 12,draw
			[English]]
		[Set 13,draw
			[French]]]{\draw[black] (.east)
node[anchor=west,align=center]{SVO};
\draw[black] (.south)--++(-7.9em,-3.5ex);}
]
{\draw[black] (.west)
node[anchor=east,align=center]{Arg Str};}
\end{forest}

\end{center}
%%%%%%%%%%%%%


%%%%%%%%%%%%%%%%%%%%%%
%%%%%%%%%%%%%%%%%%%%%%
\subsection{Forest Sets: rounded corners and labels}


\begin{center}
\begin{forest}
	for tree={rounded corners=3mm, inner ysep=.6mm,
		very thick,draw=black!50,
		top color=white,bottom color=black!20,
		font=\ttfamily, l=2cm,s sep=1cm}
	[{ ~Set 7 ~ }
	[{ ~Set 4 ~ },name=Set4
	[{ ~Set 1 ~ },name=Set1
	%			[German,no edge]
	]
	%		[{ ~Set 2 ~ },name=Set2,tier=word]
	]
	[{ ~Set 5 ~ },name=Set5
	[{ ~Set 2 ~ },name=Set2,tier=word,no edge]
	]
	[{ ~Set 6 ~ },name=Set6]
	[{ ~Set 3 ~ },name=Set3,tier=word]
	]{
		\draw[] (Set1.south)++(270:1.5em)
		node[anchor=south,align=center]{German};
		%	
		\draw[] (Set2.south)++(270:1.7em)
		node[anchor=south,align=center]{Spanish};
		%
		\draw[] (Set3.south)++(270:1.5em)
		node[anchor=south,align=center]{Korean};
		%
		\draw[black] (Set5.south)--(Set1.north);
		\draw[black] (Set5.south)--(Set3.north);
		\draw[black] (Set6.south)--(Set2.north);
		\draw[black] (Set6.south)--(Set3.north);
		\draw[black] (Set4.south)--(Set2.north);
		%\node [below=of Set1] (g){\color{white}G};
		%\node [above of=g] {German};
		%\node [below=of Set2] (s){\color{white}S};
		%\node [above of=s] {Spanish};
		%\node [below=of Set3] (k){\color{white}K};
		%\node [above of=k] {Korean};
		%\node [left=of Set8] {\begin{tabular}{@{}c@{}}Arg St\\V2\\SOV\\VC\end{tabular}};
	}
\end{forest}
\end{center}


%%%%%%%%%%%%%%%%%%%%%%
%%%%%%%%%%%%%%%%%%%%%%
\subsection{Tikz Flowchart}

\begin{tikzpicture}
	
	\node[draw, thick,
	minimum width=2cm,
	minimum height=1.5cm] at (0,0) (Block 1) {myDocument.tex};
	
	\node [below=of Block 1] (write1) {write};
	
	\node[draw, thick,
	minimum width=2cm, 
	minimum height=1.5cm] at (6.5,-2) (Block 2) {myDocument.aux};
	
	\node [above=of Block 2] (write2) {write};
	
	\node [left=of Block 2] (read) {read}; 
	
	\node[draw, thick,
	minimum width=2cm, 
	minimum height=1.5cm] at (10.2,-2) (Block 3) {myDocument.log};
	
	\node[draw, thick,
	minimum width=2cm, 
	minimum height=1.5cm] at (0,-4) (Block 4) {myDocument.pdf};
	
	\draw[-stealth,thick] (Block 1) -- (write1);
	\draw[-stealth,thick] (write1) -- (Block 4);
	\draw[-stealth,thick] (Block 1) -- (write2);
	\draw[-stealth,thick] (write2) -- (Block 2);
	\draw[-stealth,thick] (Block 2) -- (read);
	\draw[-stealth,thick] (read) -- (write1);
	\draw[-stealth,thick] (write2) -| (Block 3);

\end{tikzpicture}

\clearpage

%%%%%%%%%%%%%%%%%%%%%%
%%%%%%%%%%%%%%%%%%%%%%
\subsection{Tikz-qtree Sets}

\begin{figure}[htbp]
\centering
\begin{tikzpicture}
%%%%%%%%%%%%%
%%% Tikz Set
    \tikzset{level 1+/.style={level distance=5\baselineskip}}%
    \tikzset{sibling distance=18pt}
    \tikzset{every tree node/.style={
                          %  The shape:
                          rectangle,minimum size=6mm,rounded corners=3mm,
                          %  The rest
                          very thick,draw=black!50,
                          top color=white,bottom color=black!20,
                          font=\ttfamily},node distance=2mm}
%%%%%%%%%%%%%
    \Tree[.\node (Set3) { ~Set 3~ };
                  \node (Set1) { ~Set 1~ }; \node (Set2) { ~Set 2~ }; ]  

    \node [below=of Set1] {German}; \node [below=of Set2] {Dutch}; 

     \node [left=of Set3] {\begin{tabular}{@{}c@{}}Arg St\\V2\\SOV\\VC\end{tabular}};

\end{tikzpicture}
\caption{\label{fig-german-dutch}Common properties in German \& Dutch \citep{MuellerS14a}}
\end{figure}



%\clearpage

%%%%%%%%%%%%%%%%%%%%%%
%%%%%%%%%%%%%%%%%%%%%%
\subsection{Type hierarchy, multiple inheritance, and scalebox}


%%%%%%%%%%%%%%%%%%%%%%%%%%%%%%
%% BEGIN Figure 
\begin{figure}[h!]
	\centering
\scalebox{.7}{
	\begin{forest}
		[\emph{sem-rels}
		[\emph{$\theta$-role}
		[\emph{proto-agent}
		[\emph{agent}, name=ag
		[\emph{pure}\\ \emph{agent}]
		]
		[\emph{causer}, name=csr
		[\emph{agentive} \\ \emph{causer}, name=agcsr]
		[\emph{pure} \\ \emph{causer}]
		]
		[\textbf{\emph{stimulus}}
		[\textbf{\emph{stimulus-}}\\ \textbf{\emph{causer}}, name=stcsr]	
		[\textbf{\emph{subject}}\\ \textbf{\emph{matter}}]
		[\textbf{\emph{target}}]				
		]
		]
		[\emph{proto-patient}
		[\textbf{\emph{experiencer}}]
		[\emph{patient}]
		]	
		]
		[\emph{pred}
		[$kick$]
		[$love$]
		[\dots ]
		]
		[\dots ]
		]{
			\draw[black] (csr.south)--(stcsr.north);
			\draw[black] (ag.south)--(agcsr.north);
		}	
	\end{forest}	
}	
	\caption{Type hierarchy for \emph{semantic-relations} \citep{MyP&Co18a}}	
	\label{fig:SemRels}
\end{figure}
%% END Figure 
%%%%%%%%%%%%%%%%%%%%%%%%%%%%%%


\clearpage



%%%%%%%%%%%%%%%%%%%%%%
%%%                References                    
%%%%%%%%%%%%%%%%%%%%%%


\phantomsection	%this allows hyperlink in ToC to work

\addcontentsline{toc}{section}{References}

\bibliographystyle{enChicagoMyP}	

\bibliography{bibfile}

\end{document}