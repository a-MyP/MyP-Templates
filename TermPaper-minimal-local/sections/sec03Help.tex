%%%%%%%%%%%%%%%%%%%%%%%%%%%%
%%%%%%%%%%%%%%%%%%%%%%%%%%%%
\section{Help}
\label{sec:Help}
%%%%%%%%%%%%%%%%%%%%%%%%%%%%


%%%%%%%%%%%%%%%%%%%%%%%%%%%%
%%%%%%%%%%%%%%%%%%%%%%%%%%%%
\subsection{Own commands}
\label{ssec:Commands}
%%%%%%%%%%%%%%%%%%%%%%%%%%%%


Here you can see the result of some of the own commands (name of command in bold) defined in the file \texttt{02commands-ha}. 

%%%%%%%%%%%%%%%%%%%%%%%%%%%%
%\subsubsection{Abbreviations}

\begin{table}[h!]
	\centering
	\begin{tabular}{ll|ll}
	\multicolumn{2}{l|}{\textbf{German}} & \multicolumn{2}{l}{\textbf{English}} \\
	\textbf{input} & \textbf{output} & \textbf{input} & \textbf{output}  	\\
\midrule
	\textbackslash dash & \dash & 	\textbackslash ao & \ao \\ %%% ZG out
	\textbackslash idR & \idR  & \textbackslash cf\{page xy\} & \cf{page xy} \\ 
	\textbackslash su & \su  & \textbackslash cfe\{ex:1\} & \cfe{ex:1} \\ 
	\textbackslash ua & \ua  & \textbackslash ia & \ia \\ %%% ZG out
	\textbackslash va & \va  & \textbackslash ie & \ie \\ %%% ZG out
	\textbackslash zB & \zB & \textbackslash fe & \fe \\ %%% ZG out
	& & \textbackslash vs & \vs\\ %%% ZG out
	& & \textbackslash wrt & \wrt   %%% ZG out
	\end{tabular}
\caption{Abbreviations}
\end{table}
%%%%nur \cf{page xy} und \cfe{ex:1} nötig

%%%%%%%%%%%%%%%%%%%%%%%%%%%%	
%\subsubsection{Type setting}

\begin{table}[h!]
	\centering
	\begin{tabular}{l|l|l}
		\textbf{input} & \textbf{output} & \textbf{function} \\
		\midrule
		\textbackslash gqq\{test\} & \gqq{test} & German double quotation marks \\  
		\textbackslash gq\{test\} & \gq{test} & German single quotation marks\\ 
		\textbackslash gs\{test\} & \gs{test} & putting something between dashes \\ %%% ZG out
		\textbackslash obj\{test\} & \obj{test} & marking object language \\ 
		\textbackslash term\{test\} & \term{test} & for terminology \\ 
		\textbackslash size\{test\} & \size{test} & e.g.\ to resize quotations \\ 
		Test\textbackslash scdown\{test\} & Test\scdown{test} & marking grammatical categories \\ 
	\end{tabular}
\caption{Type setting}
\end{table}

\bigskip

%%%%%%%%%%%%%%%%%%%%%%%%%%%%	
%\subsubsection{Linguistic typography}

\begin{table}[h!]
\centering
	\begin{tabular}{l|l|l}
		\textbf{input} & \textbf{output} & \textbf{function} \\
		\midrule
		\textbackslash ra Test & \ra Test & right arrow without space \\ 
		\textbackslash ras Test & \ras Test & right arrow with space \\  %%% ZG out
		\textbackslash la Test & \la Test & left arrow without space  \\ 
		\textbackslash las Test & \las Test & left arrow with space \\ %%% ZG out
	\end{tabular}
\caption{Linguistic typography I}
\end{table}

\clearpage

\begin{table}[h!]
\centering
	
\scalebox{.85}{
\begin{tabular}{l|l|p{7.5cm}}
	\textbf{input} & \textbf{output} & \textbf{function} \\
	\midrule
		\textbackslash ab\{ch\} &\ab{ch}& notation for features and graphemes \\ %%% ZG out
		\textbackslash abe\{test\} & \abe{test} & for features and graphemes in italics \\ %%% ZG out
		\textbackslash type\{e,t\} & \type{e,t} &  for single types \\ 
		\$\textbackslash typem\{e,t\textbackslash typem\{e,t\}\}\$ & $\typem{e,t\typem{e,t}}$ &  for complex types in math mode \\ 
		\$\textbackslash ds\{test\}\$ & $\ds{test}$ & for defining denotational sets: $R \in $ $\ds{\typem{e, \typem{e,t}}}$ \\ 
		\textbackslash sem\{test\} & \sem{test} & meaning brackets \\ 
		\$\textbackslash semm\{test\}\$ & $\semm{test}$ & meaning brackets in math mode \\		\textbackslash pred\{test\} &  \pred{test} & for differentiating between expressions in object language and predicates in formulae: $\semm{sleep} := \lambda x . \pred{sleep} (x)$\\ 
		\textbackslash predO\{test\} & \predO{test}  & for operators: $\lambda x \lambda P . \predO{beg}(P(x))$ \\  
		\textbackslash val\{neut\} & \val{neut} & for HPSG values\\ %%% ZG out: wo sind Unterschiede zu \obj{test}
		\textbackslash feat\{gender\} & \feat{gender} & for HPSG features \\ %%% ZG out: wichtig wo sind Unterschiede zu \term{test}
		&& \\
	\midrule
	&&\\
		\textbackslash down\{test\}text & \down{test} text & subscript \\
		\textbackslash up\{test\}text & \up{test} text & superscript \\
		\$\textbackslash downm\{test\}text\$ & $\downm{test}text$  & subscript normal font in math mode\\
		\$\textbackslash upm\{test\}text\$ & $\upm{test}text$   & superscript normal font in math mode\\
		\end{tabular}
}
	\caption{Linguistic typography II}
\end{table}


%\bigskip

%%%%%%%%%%%%%%%%%%%%%%%%%%%%	
\subsubsection{X-bar notation}

\begin{multicols}{2}


\begin{tabular}{l|l}
	\textbf{input} & \textbf{output} \\
	\midrule
\textbackslash xzero\{X\} & \xzero{X} \\
\textbackslash xprime\{X\} & \xprime{X} \\
\textbackslash xxprime\{X\} &\xxprime{X} \\
\textbackslash xxxprime\{X\} &\xxxprime{X} \\
\textbackslash maxbar\{X\} & \maxbar{X} \\
\end{tabular}
\captionof{table}{Notation for normal text}

\begin{tabular}{l|l}
		\textbf{input} & \textbf{output} \\
	\midrule
\textbackslash ezerobar\{X\} & \ezerobar{X} \\ %%% ZG out: kann man mut \textit{\xxxprime{X}}
\textbackslash exprime\{X\} & \exprime{X} \\ %%% ZG out
\textbackslash exxprime\{X\} & \exxprime{X} \\ %%% ZG out
\textbackslash exxxprime\{X\} & \emph{\exxxprime{X}} \\ %%% ZG out
\textbackslash emaxbar\{X\} & \emaxbar{X} \\ %%% ZG out
\end{tabular}
\captionof{table}{Notation in italics}


\end{multicols}

\begin{table}[h!]
\centering
\begin{tabular}{l|l}	
\textbf{input} & \textbf{output} \\
	\midrule
\textbackslash eibar\{X\} & \eibar{X} \\ %%% ZG out
\textbackslash eiibar\{X\} & \eiibar{X} \\ %%% ZG out
\$\textbackslash overbar\{X\}\$ in math mode & $\overbar{X}$ \\ %%% ZG out: kann immer mit \textit{\iibar{X}}
\end{tabular}
\caption{Notation for the \texttt{gb4e} commands, but in italics}
%\captionof{table}{Notation for the \texttt{gb4e} commands, but in italics}
\end{table}

\clearpage

\begin{itemize}

\item  The package \texttt{lsp-gb4eMyP} already provides the following commands which are therefore commented out in the commands file:
\begin{table}[h!]
	\centering
\begin{tabular}{l|l}
	\textbf{input} & \textbf{output} \\
	\midrule
	\textbackslash obar\{X\} & \obar{X}\\
	\textbackslash ibar\{X\} & \ibar{X}\\
	\textbackslash iibar\{X\} & \iibar{X} \\
\end{tabular}
\caption{\texttt{lsp-gb4eMyP} commands}
\end{table}


\end{itemize}

%	\begin{description*}
%		\item[\textbackslash \_] 
%	\end{description*}
%%%%%%%%%%%%%%%%%%%%%%%%%%%%
\subsubsection{Colours}

\begin{table}[h!]
	\centering
	\begin{tabular}{l|l|l}
		\textbf{input} & \textbf{output} & \textbf{function} \\
		\midrule
		\textbackslash blue\{test\} & \blue{test} & blue text \\
		\textbackslash green\{test\} & \green{test} & green text \\
		\textbackslash red\{test\} & \red{test} & red text \\
		\textbackslash clrr\{test\} & \clrr{test} & red box \\
		\textbackslash clry\{test\} & \clry{test} & yellow box \\
	\end{tabular}
	\caption{Colours}
\end{table}


%%%%%%%%%%%%%%%%%%%%%%%%%%%%
\subsection{Examples}
In this document, the package \texttt{lsp-gb4eMyP} is used for creating example environments.
It is a slightly modified (almost error-free) version of \texttt{gb4e} (see the \texttt{gb4e} manual or \citet{Freitag&MyP15a}).
\texttt{lsp-gb4eMyP} can be used with the same \LaTeX \ syntax as \texttt{gb4e}:

\smallskip
\noindent
\textbf{\textbackslash begin\{exe\}}\\
\textbf{\textbackslash ex} This is an example\\
\textbf{\textbackslash ex} This is the second example.

\textbf{\textbackslash begin\{xlist\}}

\textbf{\textbackslash ex} Subordinate examples with different numbering

\textbf{\textbackslash ex} These examples have letter numbering.

\textbf{\textbackslash ex} Another example with letter numbering

\textbf{\textbackslash end\{xlist\}}\\
\textbf{\textbackslash end\{exe\}}

\clearpage

But \texttt{lsp-gb4eMyP} also provides a somewhat simpler syntax:

\smallskip

\noindent\textbf{\textbackslash ea} This is an example\\
\textbf{\textbackslash ex} This is the second example.

\textbf{\textbackslash ea} Subordinate examples with different numbering

\textbf{\textbackslash ex} Another example with letter numbering

\textbf{\textbackslash ex} These examples have letter numbering.

\textbf{\textbackslash z}\\
\textbf{\textbackslash z}

\medskip

The result of both is the same:

\ea\label{ex:1} This is an example.

\ex This is the second example.

	\ea Subordinate examples with different numbering
	
	\ex These examples have letter numbering.
	
	\ex Another example with letter numbering
	\z   
\z

%%%%%%%%%%%%%%%%%%%%%%%%%%%%
\subsection{Text positioning}

\settowidth\jamwidth{[Test jambox]}

You can't use tabulators in \LaTeX\, but there is a much neater way to position a comment or remark with certain distance to the rest of your text: the package \texttt{jambox}. It provides the command \texttt{\textbackslash jambox} whose distance from the right page margin (not in measuring units, but in letters) is set with the following command:

\medskip

\noindent \verb|\settowidth\jamwidth{[Test jambox]}| %%% ZG: nicht klar 

\smallskip

\verb|\jambox{[Test jambox]}|	\jambox{[Test jambox]}

\verb|\jambox{[Test]}|	\jambox{[Test]}

\medskip

\settowidth\jamwidth{(Chomsky, 1957: 15)}
\ea Colorless green ideas sleep furiously. \jambox{\citep[15]{Chomsky57a}}

\ex 
\gll Peter hat Marie erschrocken. \\
Peter has Mary frightened\\ \jambox{[Exp-Object verb]}
\glt `Peter has frightened Mary.' 

\ex
\gll Maria liebt Peter.\\
`Mary loves Peter.'\\ \jambox{[Exp-Subject verb]} %%% ZG out: wieso zweimal?
\z


\noindent If you want to write preliminary margin notes,\rnote{This note is red.} you can use the command \texttt{\textbackslash rnote} for a red note. %%% ZG out

The command \texttt{\textbackslash bnote} \bnote{This comment is blue.} lets you write blue notes. %%% ZG out: nicht unbedingt wichtig für HA, aber generell hilfreich 


%%%%%%%%%%%%%%%%%%%%%%%%%%%%
\subsection{Figures/Table}

There is a floating evironment for figures. It is floating but you can fix the figure on a position with the option \texttt{[h!]}. The environment is helpful to center figures by the command \texttt{\textbackslash centering} and to add captions:
\begin{itemize}
	\item[] \texttt{\textbackslash caption}[caption appearing in the list of figures] \{caption appearing under the figure\}
\end{itemize}
By using the command \texttt{\textbackslash includegraphics} from the package \texttt{graphicx} you may be able to include pictures. All you have to do is indicating the graphic's file path (see the example below).

\bigskip

{	\footnotesize
\noindent \texttt{\textbackslash begin\{figure\}[h!]\\
	\textbackslash centering\\
	\textbackslash includegraphics\{pictures/Young-frege\} \\
	\textbackslash caption[Young Frege]\{Young Frege\} \\
	\textbackslash end\{figure\} }
}
\begin{figure}[h!]
	\centering
	\includegraphics{pictures/Young-frege}
	\caption[Young Frege]{Young Frege}
\end{figure}

\clearpage

\noindent It works in the same way for tables:

\bigskip


{\footnotesize
\noindent \texttt{\textbackslash begin\{table\}[h!]\\
	\textbackslash centering\\
	\texttt begin\{tabular\}\{l|l\}\\
	Figure \& Table \textbackslash\textbackslash \\
	\textbackslash hline \\
	test \& test \\
	\textbackslash end\{tabular\}\\
	\textbackslash caption[Test table]\{Test table\} \\
	\textbackslash end\{table\} }}

\bigskip

\begin{table}[h!]
	\centering
\begin{tabular}{l|l}
	Figure & Table \\
	\hline
	test & test \\
\end{tabular}
	\caption[Text table]{Test table}
\end{table}



%%%%%%%%%%%%%%%%%%%%%%%%%%%%
%%%%%%%%%%%%%%%%%%%%%%%%%%%%
\subsection{Examples for different citation commands with \texttt{natbib}} 
\label{ssec:CitationCommands}
%%%%%%%%%%%%%%%%%%%%%%%%%%%%


Here are some examples for citation commands. You can find the IDs for every bibliography entry in the file \texttt{05literature-ha}, but they are also being suggested as soon as you type in one of the \texttt{\textbackslash cite} commands.

%%%% super hilfreich
\nocite{Nolda&Co14a}

\vspace{.5cm}

\begin{footnotesize}

\begin{tabular}{p{7cm}|p{5.7cm}}
	\textbf{input} & \textbf{output} \\
	\midrule
\verb|\citep{Fries&MyP16a} | & { \citep{Fries&MyP16a}} \\
	
\verb|\citep[cf.][4--5]{Chomsky57a}| & \citep[cf.][4--5]{Chomsky57a} \\
	
 \verb|\citet[cf.][]{Abney87a}| & \citet[cf.][]{Abney87a} \\
	
\verb|\citep[cf.][]{MuellerS99a}| & \citep[cf.][]{MuellerS99a} \\
	
\verb|\citep[56--76]{Heim&Kratzer00a}| & {\citep[56--76]{Heim&Kratzer00a}} \\
	
 \verb|\citealp[56]{MyP&Co14b}| & {\citealp[56]{MyP&Co14b} }\\
	
 \verb|\citealt[43ff]{Chomsky81b}| &{ \citealt[43ff]{Chomsky81b}} \\
	
\verb|\cf{\fe \citealt{MuellerS14a,|  &
	{(cf.\ \fe \citealt{MuellerS14a, Chomsky73a};} \\ 
\verb|Chomsky73a,Wiese&Co14a}}| &{\citealt{Wiese&Co14a})} \\
\end{tabular}

\end{footnotesize}
%\cf{\fe \citealt{MuellerS14a, Chomsky73a, Wiese&Co14a} } 


%%%%%%%%%%%%%%%%%%%%%%%%%%%%
%%%%%%%%%%%%%%%%%%%%%%%%%%%%
\subsection{Examples for different bibliographical entries}  %%% ZG out
\label{ssec:BibEntries}
%%%%%%%%%%%%%%%%%%%%%%%%%%%%


In order to see which information you need in your Bib\TeX\ file for every different entry type (article, book, manuscript, etc.), check the file: \texttt{05literature-ha}, or \citet{Freitag&MyP15a}. If you want to see the output for every specific entry type (\fe \texttt{phdthesis} \vs \texttt{book}), take a look at the bibliography of this PDF. This only works in some cases, but you can also try to hold \textsc{Ctrl} (or \textsc{Cmd} in Mac) and click on the entries' IDs. If it turns underlined and blue, you will be taken to this exact entry in the \texttt{05literature-ha} file.



\begin{itemize*}
	\item PhD Thesis: \citep{Abney87a}
	
	\item Article in an edited book: \citep{Ackema15a}
	
	\item Book: \cite{Adger04a}
	
	\item Edited book: \citep{MyP&Co14b}
	
	\item Article in a journal: \citep{Barwise&Co81a}
	
	\item Article in an online journal or database:
	\citep{Kolb&Co10a}
	
	\item Unpublished work: \citep{LeipzigGloss15a}
	
	\item Manuscript: \citep{MyP17c}
	
	\item Published work without author, using a key, \ie an abbreviation for the citation (this can be used \fe for corpora): \citep{DR17a}
	
	\item Published entry in an encyclopedia (online): \citep{MyP18b}
\end{itemize*}


%%%%%%%%%%%%%%%%%%%%%%%%%%%%
%%%%%%%%%%%%%%%%%%%%%%%%%%%%
\subsection{Helpful literature} %%% ZG out
\label{ssec:HelpLiterature}
%%%%%%%%%%%%%%%%%%%%%%%%%%%%


When writing your term paper, you can take a look at the following literature for further help (German explanations are for texts in German):
%inquire further these works:


\begin{itemize*}
	\item \citet{DR17a}: Für Fragen der Rechtschreibung 
	
	\item \citet{MyP17c} oder \citet{Rothstein11a}: Für Fragen bzgl.\ der Fertigstellung von Hausarbeiten
	
	\item \citet{Haspelmath14a}: General style rules for linguistic papers
	
	\item \citet{LeipzigGloss15a}: Glossing rules
	
	\item \citet{Freitag&MyP15a}: Für Fragen bzgl.\ \LaTeX\
	
	\item \citet{Kohm&Co13a}: Für Fragen bzgl.\ der Formatierung mit dem KOMA-Script
	
	\item \citet{Kolb&Co10a}: In case of questions regarding the syntax of \texttt{gb4e} and \texttt{lsp-gb4eMyP}
	
\end{itemize*}

